\documentclass[a4 paper,12pt]{article}\usepackage{xepersian}
\settextfont{Yas}
\setdigitfont{Yas}
\title{بِسْمِ اللَّـهِ الرَّحْمَـٰنِ الرَّحِيمِ }
\author{پروژه لاتک}
\begin{document}
\maketitle



\noindent
نام   : عذرا نام خانوادگی : حسینی          شماره دانشجویی : ۹۷۰۰۴۳۱۴۹\\  رشته : مهندسی کامپیوتر دانشگاه پیام نور پردیس  \\ درس:  روش پژوهش و ارائه  \\ استاد : جناب آقای دکتر سید علی رضوی ابراهیمی\\

\noindent
                                                                                                                                                                                                                                                                                                                                                                                                          سه صفحه ازکتاب : Strategies Methodes E-search
\vspace{0.1cm}
\vspace{0.1cm}

\begin{latin}
  \vspace{0.1cm}

\vspace{0.1cm}     
Authority\\
\vspace{0.1cm}
\vspace{0.1cm}
\vspace{0.1cm}

\noindent
Finally, e-researchers must be able to authenticate the authority  and thus attest to the reliability of the information  that they review in their research reports. The capacity for  easy  publication of  both  valid  and   invalid  information  on  the Net compels e-researchers to acquire a new set ofcritical evaluation skills. This topic is covered  in• depth in the next section\\
\vspace{0.1cm}

\noindent
EVALUATING AND AUTHENTICATING NET-BASED INFORMATION\\
\vspace{0.1cm}
\vspace{0.1cm}

\noindent
Either peer or editorial review authenticates much of the formal information that researchers use to build their literature review. The Net does not negate these valuable sources of authentication. Indeed, as noted earlier, the convenience and accessibility of the Net promises to improve both the speed and quality of the review  process. How• ever, the Net also provides opportunity for authors to directly publish their workand bypass any formof review. Of course, anyone can publish almost anything on the Net, resulting in what December (I94)  refers to as an increasing amount  of information that is a "thin soup of redundant, poor quality or incorrect information. A flood of information unfiltered by  the critical and  noise-reducing influences 
\noindent
of collaboration and peer review can overwhelm  users and obscure the value of the Web itself." Sifting through the mass of advertising material, vanity publications, and gray writings to find credible and high-quality literature is an onerous task (Harris, 1997; Smith, 1997; Till• man, 2000)-albeit, it is becoming easier to find credible and peer-reviewed publica• tions as articles from  many scholarly journals are being published online. But even within the morass of non-peer-reviewed networkeddata are valuable nuggets of infor•
mation. To guarantee the reliability and credibility of this information, specific crite•
ria  should  be  applied  in  evaluating  the  publications  found. Hallmarks of what  is
consistently considered to be valuable, credible, and high-quality information that can
be  used when  evaluating  publications found  on the Net are clustered into the cate•
gories of authority, accuracy, bias and objectivity, and coverage. Of course, determin•
ing what is  valuable,  credible,  high-quality  information  is  a slippery   subject--what
may be noteworthy in one field ofstudy may be unworthy  in another.


\vspace{0.1cm}

\noindent
Authority\\
\vspace{0.1cm}
\vspace{0.1cm}

\noindent
Although authority is not synonymous with truthfulness (if it were we would have had no reformations, revolutions,  or paradigm shifts), a  reference  linked to an institution or writings of known authority serve as useful clues that the information is likely to be reliable and accurate. The authority can be deduced from indicators in the Web page. These include the host computer, which is the URL listed under the domain name of an established authority or institution such as a known university, government, or pub• lic agency.  Other indicators include prominent links to the homepage of the corre• sponding organization, and whether you can follow  these links. When you read further information about the organization, you feel confident that the Web site author is a credible source.  For example,  are the authors' names, email addresses, and telephone numbers provided in the page so that you can seek further information about the con• tent?  Are there links to the homepages of the authors so that you can check on their career path? Can you check for any peer-reviewed publications of the authors? Are there links within the document to other sites or information  that  you can check for sources of authority?
\noindent
Accuracv\\
\vspace{0.1cm}
\vspace{0.1cm}

\noindent
Accuracy is difficult to attest, especially ifall relevant information is            not provided. In our careens as academics, we regularly review materials submitted for publication.  In the review  process, we are allowed  to request additional information (such  as background or confirming evidence) before acceptingan article for publication.  If the Net-based infor• mation is a component of a  peer-reviewed e-journal, then you have some guarantee that experts  in the  field have reviewed  the data collection process, methodology, theoretical underpinning, and analysis techniques, and thus, the probability of  curacy of results is  enhanced.  However,  if the  information  has not  been  peer-reviewed,   then  e• researchers are forced to undertake this evaluation themselves. After reviewing the research processes  described  in  this text,  you should  have  some  sens  of the issues involved in chairing accuracy ofyour own work; these same criteria can be used  to eval• uate the presentations of the work of others.  Although no guarantee, one way  to asses accuracy is        to look for the referencing ofother known works in the text. Such referenc•
ing indicates  that the author has read the works of other researchers in the field and thus
has been exposed to the isues and the accepted  methodologies in the field of inquiry.

'The assessment of acuracy requires the skills of critical thinking coupled  with a strong dose of common sense.  If you find  valuable information at a site, dig deeply within  the site and associated  links;  often you may  find further work  that has been
reviewed by  the authors or other indicators of authority that will helpyou make a val•
uation of the information's accuracy. 

Bias and Objectivity\\
\noindent
Our postmodern colleagues convince us that no information is unbiased. However, quality  Net information  is clear about the source of any  funding, organizational  bias, or political or moral agenda. If such disclosure is not apparent, and especially  if the topic has commercial, political, or religious implications, e-researchers must be vigilant to insure they are not being purposely  misinformed or being provided  with  biased information. Clues to a biased perspective  include overly  strident  language, links to sites with a known bias, absence of links or arguments from  opposing viewpoints, and the absence of any linkage to unbiased authority references\\
COVERAGE\\
\vspace{0.1cm}
As  noted earlier, the rapidly  expanding nature of networked resources makes it very
difficult to determine  when  the e-researcher  has exhaustively  reviewed  all  relevant
\noindent
materials. The impossibility  of covering everything produces  a tendency  for reference sites to tightly restrict the scope of knowledge that they attempt  to explicate. Thus, the e-researcher is forced to review and examine resources on an ever-enlarging set of so• called boutique  sites that are narrowly focused on a particular subset of information, or conversely to search through  very broad (and often shallow) overview reference sites. Our  only advice to ameliorate this problem  is to  use the search engines regularly and effectively, taking special  care to review the most timely sites.\\
\noindent
Additional  Resources\\
\noindent
There are many Net resources (mostly created by dedicated librarians) offering advice, checklists,  and methods  for asserting the authority, accuracy, and veracity of WWW• based resources. A list of a few of these resources is available at World  Wide Web Vir• tual Library at http://www.vuw.ac.nz/~agsmith/evaln/evaln.htm\\
\vspace{0.1cm}
\noindent
FINDING SOURCES OF INFORMATION FOR THE LITERATURE  REVIEW\\
The e-researcher uses the traditional  sources of relevant  literature including  library• based  books, journals, printed conference presentations, and  the  popular  press. In addition,  the e-researcher uses the growing body of literature that is available online. These sources can be divided into two groups, formal sources that are usually peer or professionally reviewed and informal  resources  that are gathered through  discussion groups, conference presentations, and private correspondence.
Fonnal Online  Resources\\
\noindent
The skilled e-researcher uses a variety of search engines and searching techniques  to scour  the  Internet for relevant  materials.  Often  such  searches begin  with  keyword searches in popular  general-purpose search engines (such as Google.com)  or scrolling down a hierarchical subject listing of sites (such as Yahoo!). The techniques  for search• ing that are detailed in Chapter 2 will be useful in finding relevant information.
Online literature searches often progress to  the searching of specific databases or Web indexes that focus on a particular field of study or topic. These so-called boutique search engines  exclude  references to  unrelated topics. For  example,  an index  that excludes all references to non-academic topics such as sports, gambling, television, and sex will eliminate large numbers of sites of little  or no  interest to an e-researcher. Examples  of  these  specialized  search  engines  include  ProFusion  (http://beta.profusion.com/), which provides vertical searches on ten different subject areas includ• ing arts and humanities; All Academic (http://www.allacademic.com/), which provides free searching and displays the full text of academic journal  articles and book chapters; and an even more  focused  service,  Education-line (http://www.leeds.ac.uk/educol/), 



\end{latin}
\noindent



\end{document}

